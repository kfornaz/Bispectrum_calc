Toda a solução foi pensada para utlizar multiplas threads de Python

Para o calculo do bell vc pode fazer como os exemplos abaixo:

python execBat.py 0 10 (calcula i de 0 a 10)

python execBat.py 0 100 (calcula i de 0 a 100)

python execBat.py 0 500 (calcula i de 0 a 500)


Por que fazer a seguinte sequencia?

python execBat.py 0 100

python execBat.py 100 200

python execBat.py 200 300

python execBat.py 300 400

python execBat.py 400 500

para dividir o processamento em diversas threads e janelas. Dessa maneira não precismos aguardar o i=200 para processarmos i=201. Assim, podemos executar essa operaçẽos em paralelo. 

Podemos rodar:

python execBat.py 0 5, 
python execBat.py 5 10  

etc. 

Dessa forma criaremos lotes de 5 arquivos otimizando ainda mais a execução. 
Para isso teriamos que fazer alguns ajustes na solução atual (estamos providenciando estes ajustes que implantaram o conceito de passo e ciclo na paralelização). 
O mesmo procedimento se aplica ao calculo das médias/devio padrão e criação dos plots conforme indicado abaixo



python calcMedia.py 0 10 B_ell mean.csv

python calcMedia.py 0 100 B_ell mean.csv

python calcMedia.py 0 500 B_ell mean.csv



python plotMapBat.py B_ell 0 10

python plotMapBat.py B_ell 0 100

python plotMapBat.py B_ell 0 500